\documentclass[sans,mathserif,aspectratio=169, 10pt]{beamer}

\usepackage{booktabs}
\usepackage[spanish, mexico]{babel}
\selectlanguage{spanish}
\decimalpoint
\usepackage[utf8]{inputenc}
\usepackage{fourier}
\newcommand{\quotes}[1]{``#1''}
\usepackage{epstopdf}
\usepackage{mathtools}
\DeclarePairedDelimiter{\ceil}{\lceil}{\rceil}
\usepackage{commath}
\usepackage{amsmath,tabularx}
\usepackage{listings}
\usepackage{xcolor}
\usepackage[edges]{forest}
\graphicspath{{Images/}}
\usepackage[mathscr]{euscript}
\newcommand{\overbar}[1]{\mkern 1.5mu\overline{\mkern-1.5mu#1\mkern-1.5mu}\mkern 1.5mu}
\newcommand*\mean[1]{\overbar{#1}}

\newcommand\Fontvi{\fontsize{9}{7.2}\selectfont}
\definecolor{foldercolor}{RGB}{124,166,198}

%Tikz stuff
\usepackage{tikz}
\usetikzlibrary{trees}
\usetikzlibrary{shapes,arrows,positioning}
\usepackage{forest}

\colorlet{punct}{red!60!black}
\definecolor{background}{HTML}{EEEEEE}
\definecolor{delim}{RGB}{20,105,176}
\colorlet{numb}{magenta!60!black}

\lstdefinelanguage{json}{
    basicstyle=\normalfont\ttfamily\tiny,
    columns=flexible,
    keepspaces=true,
    frame=lines,
    backgroundcolor=\color{background}
}

%Forest Set
\tikzset{
    invisible/.style={opacity=0,text opacity=0},
    visible on/.style={alt=#1{}{invisible}},
    alt/.code args={<#1>#2#3}{%
      \alt<#1>{\pgfkeysalso{#2}}{\pgfkeysalso{#3}} % \pgfkeysalso doesn't change the path
    },
}
\forestset{
  visible on/.style={
    for tree={
      /tikz/visible on={#1},
      edge={/tikz/visible on={#1}}}}}

\mode<presentation>

%Colors
\definecolor{steel}{RGB}{52,102,136}
\definecolor{moss}{RGB}{139,187,159}
\definecolor{burnt}{RGB}{187,102,65}
\definecolor{sandy}{RGB}{186, 168, 111}
\definecolor{RoyalBlue}{RGB}{0,35,102}
\definecolor{cream}{RGB}{254, 246, 235}
\definecolor{slate}{RGB}{82, 85, 100}
\definecolor{fall}{RGB}{194, 91, 86} % frame color
\definecolor{light}{RGB}{150, 192, 206}

% Tikz Image Extras
%Colors
\definecolor{buttercup}{RGB}{245,204,93}
\definecolor{calypso}{RGB}{51,102,136}
\definecolor{casper}{RGB}{170,196,209}
\definecolor{beige}{RGB}{254,250,225}
\definecolor{grenadier}{RGB}{193,49,0}
\definecolor{chiffon}{RGB}{255,251,208}
\definecolor{tawny}{RGB}{204,102,0}
\definecolor{danube}{RGB}{102,153,204}
\definecolor{rich}{RGB}{162,95,8}
\definecolor{york}{RGB}{122,186,122}
\definecolor{royals}{RGB}{18,55,143}
\definecolor{powder}{RGB}{96,164,223}
\definecolor{gold}{RGB}{247,204,019}
\definecolor{columbia}{RGB}{117,178,221}
\definecolor{crater}{RGB}{179,53,18}

\definecolor{grey}{RGB}{168,168,168}
\definecolor{chair}{RGB}{179,133,65}
\definecolor{navy}{RGB}{9,40,105}

%Tikz Styles
\tikzstyle{mybox} = [draw=grenadier, fill=chiffon, very thick, dashed, rectangle, rounded corners, 
    inner sep=10pt]
\tikzstyle{title} = [fill=grenadier, text=white]
\tikzstyle{program} = [rectangle, rounded corners, fill=grenadier!75, text centered, text width=5em,inner sep=10pt]
\tikzstyle{external} = [rectangle, rounded corners, fill=columbia, text centered, text width=5em,inner sep=10pt]
\tikzstyle{tool} = [rectangle, rounded corners, fill=york, text centered, text width=7em,inner sep=10pt]
\tikzstyle{input} = [rectangle, rounded corners, fill=danube, text centered, text width=5em]
\tikzstyle{proposal} = [rectangle, very thick, dashed, fill=grenadier!95, text centered]
\tikzstyle{line} = [draw, -latex']
    

%Structure
\usefonttheme{professionalfonts}
\hypersetup{colorlinks,linkcolor=fall!80,urlcolor=fall!80}
\setbeamercolor{local structure}{fg=fall}
\setbeamercolor{background canvas}{bg=cream}
\setbeamercolor{frametitle}{bg=fall,fg=cream}
\setbeamercolor{title}{fg=steel!115}
\setbeamercolor{author}{fg=fall}

\defbeamertemplate*{title page}{customized}[1][]
{\centering
  \usebeamerfont{subtitle}\usebeamercolor[fg]{subtitle}\insertsubtitle\par
  \usebeamerfont{title}\usebeamercolor[fg]{title}{\bfseries\inserttitle}\par
  \vfill
  \usebeamerfont{author}\usebeamercolor[fg]{author}\insertauthor\par
  \vfill
  \usebeamerfont{institute}\insertinstitute\par
  \usebeamerfont{date}\insertdate\par
  \usebeamercolor[fg]{titlegraphic}\inserttitlegraphic
}

% Progressbar
\usepackage{tikz}
\usetikzlibrary{calc}

\makeatletter
\def\progressbar@progressbar{} % the progress bar
\newcount\progressbar@tmpcounta% auxiliary counter
\newcount\progressbar@tmpcountb% auxiliary counter
\newdimen\progressbar@pbht %progressbar height
\newdimen\progressbar@pbwd %progressbar width
\newdimen\progressbar@tmpdim % auxiliary dimension

\progressbar@pbwd=\paperwidth
\progressbar@pbht=2pt

\def\progressbar@progressbar{%

\progressbar@tmpcounta=\insertframenumber
\progressbar@tmpcountb=\inserttotalframenumber
\progressbar@tmpdim=\progressbar@pbwd
\multiply\progressbar@tmpdim by \progressbar@tmpcounta
\divide\progressbar@tmpdim by \progressbar@tmpcountb

  \begin{tikzpicture}[very thin]

  \shade[draw=steel!115,top color=steel,bottom color=steel,middle color=steel!115] %
    (0pt, 0pt) rectangle ++ (\progressbar@tmpdim, \progressbar@pbht);

  \end{tikzpicture}%
 }

\addtobeamertemplate{frametitle}{}
{%
  \vspace*{-16pt}
  \begin{beamercolorbox}[wd=\paperwidth,ht=1pt,dp=1pt]{}%
    \progressbar@progressbar%
  \end{beamercolorbox}%
}%
\makeatother

%Title
\title{Development of an Improved Subgroup Method for Resonance Calculations}
\subtitle{Thesis Project Progress Report}
\author[Guillermo Ibarra]{Guillermo Ibarra\\{\small Supervised by: Dr Gustavo Alonso Vargas}}
\date{Nuclear Engineering Research Seminar, June 9th, 2020}

\definecolor{keywords}{RGB}{255,0,90}
\definecolor{comments}{RGB}{0,0,113}
\definecolor{red}{RGB}{160,0,0}
\definecolor{green}{RGB}{0,150,0}

\setbeamercovered{transparent}
\setbeamercovered{%
  again covered={\opaqueness<1->{40}}}
\beamertemplatenavigationsymbolsempty

\begin{document}

%slide
\begin{frame}
\titlepage
\end{frame}

%slide
\begin{frame}{General Objective}
\centering
\LARGE
\emph{\color{steel!115}Develop a lattice code for high fidelity analysis.}
\end{frame}

%slide
\begin{frame}{Legacy Nuclear Reactor Analysis Procedure}
\begin{center}
\begin{tikzpicture}[scale=0.8]
%step 1
\node<1-> [external] at (0,0) (genLIB){General $\sigma$ XS Library};

%step 2
\node<2-> [external] at (8,0) (xsLIB){Working $\sigma$ XS Library};
\path<2-> [line,crater,very thick] (genLIB) -- (xsLIB);

%step 3
\node<3-> [program] at (8,-4) (core){Core Analysis};
\path<3-> [line,crater,very thick] (xsLIB) -- (core);
\end{tikzpicture}
\end{center}
\end{frame}

%slide
\begin{frame}{Less General Objective}
Develop a resonance calculation methodology capable of considering:
\begin{itemize}[<+->]
\item Spatial self-shiedling effects,
\item Resonance interference,
\item Non-uniform temperature effects, and
\item Self-shielding effects of cladding isotopes. 
\end{itemize}
\end{frame}

%slide
\begin{frame}{Research Questions}
\begin{itemize}[<+->]
\item Is the Subgroup method the \emph{best} resonance calculation methodology?
\item How much \emph{accuracy} is required? 
\item What is the trade off between computational resources and accuracy?
\end{itemize}
\end{frame}

%slide
\begin{frame}{Starting Point and Proposed Road Map}
Gemma neutron transport code:
\begin{itemize}[<+->]
\item 2D method of characteristics
\item Potential improvements: CMFD, linear source approximation, hardware acceleration, random ray tracing
\end{itemize}
\pause
Proposed work outline:
\begin{enumerate}[<+->]
\item Incorporate a \emph{workhorse} equivalence method (ie WIMS)
\item More exact equivalence method, pointwise energy slowing down (Choi et al 2017)
\item Incorporation of a \emph{basic} subgroup method then add improvements
\end{enumerate}
\end{frame}

%slide
\begin{frame}{Spring 2020 Semester Overview}
\begin{enumerate}[<+->]
\item Literary review and planning process.
\item Gemma conditioning.  
\end{enumerate}
\end{frame}

%slide
\begin{frame}{Gemma Conditioning}
\begin{enumerate}[<+->]
\item Re-code in a \emph{TDD} style.
\item Code Documentation
\end{enumerate}
\end{frame}

%slide
\begin{frame}{More Specific Plans for Fall 2020}
\begin{enumerate}[<+->]
\item Module to read from a \emph{general $\sigma$ cross section library}
\item Equivalence theory resonance calculation module. 
\end{enumerate}
\end{frame}

%slide
\begin{frame}{WIMS Resonance Treatment}
\begin{enumerate}[<+->]
\item Based on IR approximation and equivalence theory.
\item Resonance integrals are tabulated as a function of temperatura and $\sigma_0$
\item Heterogeneous media are a function of escape probability.
\end{enumerate}
\end{frame}

%slide
\begin{frame}{WIMS Resonance Treatment}
\begin{enumerate}[<+->]
\item Based on IR approximation and equivalence theory.
\item Resonance integrals are tabulated as a function of temperatura and $\sigma_0$
\item Heterogeneous media are a function of escape probability.
\end{enumerate}
\pause
\begin{equation}
\sigma_x (T, \sigma_b) \approx \frac{I_x(T, \sigma_b)}{1 - \frac{I_a (T, \sigma_b)}{\sigma_b}}
\end{equation}
\end{frame}

%slide
\begin{frame}{Looking Ahead to PhD Pre-Defense}
Work within $\sim18$ months and code review article:
\begin{enumerate}[<+->]
\item Equivalence IR approximation,
\item \emph{More accurate} Equivalence theory,
\item \emph{Basic} subgroup resonance methodology 
\end{enumerate}
\end{frame}

%slide
\begin{frame}
\centering
\Huge
Thanks! \\
Questions?
\end{frame}


\end{document}
